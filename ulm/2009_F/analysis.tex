\documentclass{article}
\usepackage{amsmath}
\usepackage{amssymb}

\begin{document}

\title{Analysis of problem ULM 2009 F --- Food portion size}
\date{}
\author{Tiago Royer}
\maketitle

The problem gives $n$ integers $y_1, y_2, ..., y_n$,
representing how much each student eats,
and two special parameters $a$ and $b$.
The problem asks to minimize both the amount of food wasted
and the number of times the student has to fetch food,
weighted by the parameters $a$ and $b$,
under the restriction that no student must have to fetch more than $3$ times.

First, let's create a formula $c(S)$,
which will tell the weighted average cost of wastes and fetches,
given a portion size $S$.
The constraint is simple to formulate;
three portions must be enough to serve all students,
so $S \geq y_i/3$ for all $i$.
If $Y$ is the maximum of all $y_i$, the problem asks for
\begin{equation*}
    \min_{S \geq \frac Y 3} c(S).
\end{equation*}

\newcommand\fetches{\left\lceil \frac{y_i}{S} \right\rceil}

The student $i$ will have to fetch food $\fetches$ times;
summing from $i = 1$ to~$n$ gives $y$, the total number of fetches.

Each time, the student $i$ will pick $S$ units of food;
since only $y_i$ units will be eaten by student $i$,
the wasted food is
\begin{equation*}
    S \fetches - y_i.
\end{equation*}
Summing from $i = 1$ to $n$ gives the total wasted food.

Therefore, our desired formula is
\begin{align*}
    c(S) = ax + by &=
        a\left(
            \sum_{i=1}^n S \fetches - y_i
        \right)
        +b\left(
            \sum_{i=1}^n \left\lceil \frac{y_i}{S} \right\rceil
        \right) \\
    &= (aS+b)\sum_{i=1}^n \fetches - a \sum_{i=1}^n y_i
\end{align*}

The second term,
$a\sum_{i=1}^n y_i$,
does not depend on $S$,
so we cannot improve it.
The first term is the product of $aS + b$,
which is a strictly increasing function
(because $a \geq 1$),
and $\sum_{i=1}^n \fetches$,
which decreases as $S$ becomes larger.

However, the summation terms are ceilings of functions,
which decreases in discrete steps.
Thus, there are intervals $[i, j)$
in which none of that ceilings change.
On this interval, the cost function $c$ is affine and increasing,
and thus has minimum value $c(i)$.
Therefore, $c$ can only attain a minimum value
in a point of discontinuity of $\sum_{i=1}^n \fetches$.

Any discontinuity point of $\fetches$
is a candidate for a discontinuity point of the summation.
This function is discontinuous exactly when $y_i/S$ is an integer.
Since the problem imposes the constraint that $\fetches$
must be $1$, $2$ or~$3$,
we have at most three discontinuity points that satisfies the constraints.
Therefore, the set of discontinuity points is \emph{finite},
and have at most $3n$ points.
A brute-force approach costs thus $O(n^2)$,
which is feasible as the problem guarantees $n \leq 10^3$.

We can simplify the code by avoiding rational arithmetic.
$y_i/S = k$ is the same as $S = y_i/k$;
since all $y_i$ are integers,
the only possible portion sizes have denominator $1$, $2$ or~$3$.
Therefore, every rational number in the calculation
will have denominator $1$, $2$, $3$ or~$6$.
Thus, if we multiply everything by $6$,
we will not have to deal with fractions in most of the code.

We will multiply both $S$ and all the $y_i$ by $6$;
let's call them $S'$ and $y_i'$.
We have to rewrite $\fetches$ and $c$ in terms of these.
\begin{equation*}
    \fetches = \left\lceil \frac{6 y_i}{6 S} \right\rceil
        = \left\lceil \frac{y_i'}{S'} \right\rceil,
\end{equation*}
so we may left $\fetches$ untouched.
For $c$, we will create a $C$ function defined by $C(S) = 6c(S)$.
We have
\begin{align*}
    C(S) = 6c(S) &= 6(aS+b)\sum_{i=1}^n \fetches - 6a \sum_{i=1}^n y_i \\
        &= (aS'+6b)\sum_{i=1}^n \fetches - a \sum_{i=1}^n y_i',
\end{align*}
so we need just to multiply $b$ by~$6$
in the function that otherwise computes $c$.

\end{document}
